\documentclass{article}

% Language setting
% Replace `english' with e.g. `spanish' to change the document language
\usepackage[ukrainian]{babel}

% Set page size and margins
% Replace `letterpaper' with `a4paper' for UK/EU standard size
\usepackage[letterpaper,top=2cm,bottom=2cm,left=3cm,right=3cm,marginparwidth=1.75cm]{geometry}

% Useful packages
\usepackage{amsmath}
\usepackage{graphicx}
\usepackage[colorlinks=true, allcolors=blue]{hyperref}


\usepackage{listings}
\usepackage{color}

\definecolor{dkgreen}{rgb}{0,0.6,0}
\definecolor{gray}{rgb}{0.5,0.5,0.5}
\definecolor{mauve}{rgb}{0.58,0,0.82}


\renewcommand{\labelenumi}{\arabic{enumi})}
\renewcommand{\labelenumii}{\arabic{enumii}.}




\begin{document}


\begin{titlepage}
    \begin{center}
        \vspace*{1cm}
\Large
            Національний Технічний Університет України\\
“Київський Політехнічний Інститут”\\
Фізико-Технічний Інститут\\
  \vspace{1.5cm}
        \huge
        \textbf{СПЕЦІАЛЬНІ РОЗДІЛИ ОБЧИСЛЮВАЛЬНОЇ МАТЕМАТИКИ}
            
        \vspace{0.5cm}
        \LARGE
        КОМП’ЮТЕРНИЙ ПРАКТИКУМ №1

        Багаторозрядна арифметика
            
        \vspace{1.5cm}
            

        \vfill
            
	\end{center}
            	
	\begin{flushright}
        \Large
        	Виконав студент 3-го курсу\\
	групи ФІ-13\\
	Мельник Євгеній\\
    	\end{flushright} 
 \vspace{2.5cm}
\centering \Large Київ 2023
\end{titlepage}


\raggedright
\section{Вступ}

Звіт з першого комп'ютерного практикума по спеціальним розділам обчислюванної математики. Увесь код можна знайти за посиланням на \href{https://github.com/avept/Long-Arithmetic}{GitHub}.

\subsection{Мета}
Отримання практичних навичок програмної реалізації багаторозрядної арифметики; ознайомлення з прийомами ефективної реалізації критичних по часу ділянок програмного коду та методами оцінки їх ефективності.

\subsection{Завдання}
\begin{enumerate}
    
\item Повинні бути реалізовані такі операції:

\begin{enumerate}
  \item переведення малих констант у формат великого числа (зокрема, 0 та 1);

  \item додавання чисел;

  \item віднімання чисел;

  \item множення чисел, піднесення чисел до квадрату;

  \item ділення чисел, знаходження остачі від ділення;

  \item піднесення числа до багаторозрядного степеня;

  \item конвертування (переведення) числа в символьну строку та обернене перетворення символьної строки у число; обов'язкова підтримка шістнадцяткового представлення, бажана десяткового та двійкового.

  \item визначення номеру старшого ненульового біта числа;

  \item бітові зсуви (вправо та вліво), які відповідають діленню та множенню на степені двійки.

\end{enumerate}

    \item Проконтролювати коректність реалізації алгоритмів; наприклад, для декількох багаторозрядних $a, b, c, n$ перевірити тотожності:

\begin{enumerate}
  \item $(a+b) \cdot c=c \cdot(a+b)=a \cdot c+b \cdot c ;$

  \item $n \cdot a=\underbrace{a+a+\ldots+a}_{n}$, де $n$ повинно бути не менш за 100 ;

\end{enumerate}

і таке інше.

Продумати та реалізувати свої тести на коректність.

    \item Обчислити середній час виконання реалізованих арифметичних операцій. Підрахувати кількість тактів процесора (або інших одиниць виміру часу) на кожну операцію. Результати подати у вигляді таблиць або діаграм.

\end{enumerate}
\section{Реалізація завдання}

\subsection{Структура проекту}

Проект складається з файлів BigInt.*, та системи збірки - CMake. \\
BigInt - бібліотека довгих чисел.
У бібліотеці є клас BigInt - самі довгі числа, в якому реалізовані усі допоможні функції, арифметика,
конвертори з різних систем числення, та інше.

\subsection{Середній час виконання}
Розглянемо середній час виконання арифметичних операцій. Заміри для всіх операцій, окрім піднесення до степіню проводились на різних розмірах великого числа(на розмірах вбудованного масива від 64 до 8192) у тактах процесора. Як можна буде побачити, час виконання суттєво залежить від того, яку ми поставили можливу довжину числа (тобто довжини вбудованого масива).

Кратке уточнення: для майже всіх подальших таблиць Small numbers та Big numbers = числа довжини (Array Size)/7 та (ArraySize)*6 у шістнацятирічній системі числення. При чому варто зазначити, що судячи з вимірів, які були зроблені на випадкових числах по 1000 разів на довжинах до 1024, і по 100 разів у подальшому, можна зробити висновок що заповненність самого масиву не грає ролі у кількості тактів процесора.
\begin{enumerate}
    \item Додавання: як можна побачити з таблиці~\ref{tab:add}, кількість тактів процесора для додавання росте з збільшенням масиву. Чомусь для маленьких чисел додавання працює довше.
\begin{table}[ht!]
\centering
\begin{tabular}{|c|c|c|c|c|c|c|c|c|c|}
\hline
Array size& 64 & 128 & 256 & 512 & 1024 & 2048 & 4096 & 8192 \\\hline
Small numbers& 28 & 53 & 99 & 149 & 321 & 565 & 1133 & 2898 \\ 
Big numbers& 22 & 41 & 89 & 143 & 362 & 677 & 1269 & 2456 \\ \hline
\end{tabular}
\caption{\label{tab:add}Кількість тактів для додавання}
\end{table}
    \item Віднімання: все так само як і з додаванням, числа виходять майже однаковими. Але немає дивної поведінки, тому що для менших чисел віднімання працює швидше.
\begin{table}[ht!]
\centering
\begin{tabular}{|c|c|c|c|c|c|c|c|c|c|}
\hline
Array size& 64 & 128 & 256 & 512 & 1024 & 2048 & 4096 & 8192 \\\hline
Small numbers & 25 & 47 & 96 & 155 & 299 & 600 & 1160 & 2452 \\ 
Big numbers & 27 & 47 & 100 & 175 & 340 & 701 & 1440 & 2996 \\ \hline
\end{tabular}
\caption{\label{tab:sub}Кількість тактів для віднімання}
\end{table}



\item Множення: як можемо бачити, кількість тактів суттєво більша ніж у додавання і множення. Можливо із-за того що у множенні використовується додавання, але для деяких менших чисел алгоритм працює довше.

\begin{table}[ht!]
\centering
\begin{tabular}{|c|c|c|c|c|c|c|c|c|c|}
\hline
Array size& 64 & 128 & 256 & 512 & 1024 & 2048 & 4096 & 8192 \\\hline
Small numbers & 2540 & 9638 & 42947 & 151549 & 607967 & 2328355 & 9307792 & 38913218  \\ 
Big numbers & 2461 & 9486 & 44913 & 145122 & 587425 & 2312513 & 9437713 & 40168797  \\ \hline
\end{tabular}
\caption{\label{tab:mult}Кількість тактів для множення}
\end{table}

\item Ділення: побачивши результати, я думав що я щось переплутав, але ні. З довжинни масиву 2048 ділення починає працювати \textbf{\textit{швидше}} ніж множення.
\begin{table}[ht!]
\centering
\begin{tabular}{|c|c|c|c|c|c|c|c|c|c|}
\hline
Array size& 64 & 128 & 256 & 512 & 1024 & 2048 & 4096 & 8192 \\\hline
Small numbers & 60315 & 122052 & 262816 & 496024 & 966243 & 1895101 & 3772466 & 7639897  \\ 
Big numbers & 51360 & 101949 & 235052 & 397795 & 792811 & 1582316 & 3223783 & 6939508  \\ \hline
\end{tabular}
\caption{\label{tab:div}Кількість тактів для ділення}
\end{table}
\item Модуль числа: Працює трошки швидше ніж ділення - але не набагато, тому що використовується одна й та ж сама функція, але не обчислюється дільник, тільки остача.

\begin{table}[ht!]
\centering
\begin{tabular}{|c|c|c|c|c|c|c|c|c|c|}
\hline
Array size& 64 & 128 & 256 & 512 & 1024 & 2048 & 4096 & 8192 \\\hline
Small numbers & 41833 & 86172 & 187056 & 342929 & 681078 & 1329635 & 2651094 & 5353054  \\ 
Big numbers & 33228 & 66216 & 148376 & 255733 & 515606 & 1022484 & 2073757 & 4490010  \\ \hline
\end{tabular}
\caption{\label{tab:mod}Кількість тактів для модуля}
\end{table}


\item Піднесення в степінь: Бралась окрема вибірка з числом розміру (Array size)$*4$, і підносилась до степіню 1000. Числа з розміром масиву 8192 не брались, тому що занадто довго обраховується. :)

\begin{table}[ht!]
\centering
\begin{tabular}{|c|c|c|c|c|c|c|c|}
\hline
Array size& 64 & 128 & 256 & 512 & 1024 & 2048 & 4096  \\\hline
Small numbers & 53422 & 203852 & 634750 & 2271761 & 10142121 & 36631209 & 174713669  \\ \hline
\end{tabular}
\caption{\label{tab:pow}Кількість тактів для піднесення у степінь}



\end{table}

\end{enumerate}

\subsection{Декілька прикладів обчислення великих чисел}

Усі числа будуть у hex, тому що це саме зручне представлення. (Відображається трошки криво, тому що латех не справляється з довгими словами)
\begin{enumerate}
\item Перше число: A := 77f612aec34ab3905fdca712aecde12394834; Друге число: B := b159b7283e51f8bec6a01189c6735334

$A + B$ = 77f61dc45ebd37757f68937cafe67d8ac9b68

$A - B$ = 77f6079927d82fab4050baa8adb544bc5f500

$A * B$ = 77f6079927d82fab4050baa8adb544bc5f500531b 2d4b16ca65fa2de787cb8cc04874b8e57a7d5ad57c7796396f851b26 bed678690

$A / B$ = ad290

$A \% B$ = a8a3bbb53569d14ab95c4a129711d2f4

$A ^{14}$ = 19cfbc97fb4428f247570b0cc7a51171e3005c51a3b8e84bdd1def76c81daf58311 0cd113fd4ef4a6c98f68f487b4cf0dee4289b88491de7e00d32852b4ed8a648c9cc69336f680ad f9c212a821737f01af799a4218a5680e9aef7c9162c4fe38ef532b9054340541e8250a05848b4c 500a8d4cbfeca439b0f78471e46052a59f8e0a3444a40601fad7fbf9c50cc7a330dfad6e11f73b09 4218d929276395c1a959a6c9ba6e4da7af1109e8ac66d3f41e84b2058bde32728625649ebdf90 de20afc52d129c86b795a9ba156c93b304dc90ba78a39609bef50f2e6b773d37d2e9189243942 add65db6db92c6c1ccabbaeb363535d70a782063683adcb99d90000000

\item Перше число: A := d8096aa33c5e6; Друге число: B := ff135f6c6

$A + B$ = d80a69b69bcac

$A - B$ = d8086b8fdcf20

$A * B$ = d8086b8fdcf20d741ba7276ce94e2d413e4

$A / B$ = 165

$A \% B$ = a3f643a46263c2e

$A ^{13}$ = 1c2ebcb56b2114bda171e1f8139302063d1defed32f591601df211 4e7442aaa4757d75638b61eadb8129d07ecfb3275ab7fb5413545d06c79663897 2ecd08c3b317981d0a54b5df8172986c3ba9cd41e952206000

\end{enumerate}
\section{Висновок}
	У підсумку цього комп'ютерного практикума, я зрозумів як працювати з багаторозрядними числами з основою $2^n$, як реалізовуються операції над такими числами. Також було зробленно такий висновок - додавання, віднімання працюють найшвидше, а піднесення до степіню (доволі великого) - саме найдовше. 

\end{document}